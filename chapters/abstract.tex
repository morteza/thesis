\pagenumbering{roman}

\section*{Abstract}
 
Cognitive control is essential to human cognitive functioning as it allows us to adapt and respond to a wide range of situations and environments. The possibility to enhance cognitive control in a way that transfers to real life situations could greatly benefit individuals and society. However, the lack of a formal, quantitative definition of cognitive control has limited progress in developing effective cognitive control training programs. To address this issue, the first part of the thesis focuses on gaining clarity on what cognitive control is and how to measure it. This is accomplished through a large-scale text analysis that integrates cognitive control tasks and related constructs into a cohesive knowledge graph. This knowledge graph provides a more quantitative definition of cognitive control based on previous research, which can be used to guide future research. The second part of the thesis aims at furthering a computational understanding of cognitive control, in particular to study what features of the task (i.e., the environment) and what features of the cognitive system (i.e., the agent) determine cognitive control, its functioning, and generalization. The thesis first presents CogEnv, a virtual cognitive assessment environment where artificial agents (e.g., reinforcement learning agents) can be directly compared to humans in a variety of cognitive tests. It then presents CogPonder, a novel computational method for general cognitive control that is relevant for research on both humans and artificial agents. The proposed framework is a flexible, differentiable end-to-end deep learning model that separates the act of control from the controlled act, and can be trained to perform the same cognitive tests that are used in cognitive psychology to assess humans. Together, the proposed cognitive environment and agent architecture offer unique new opportunities to enable and accelerate the study of human and artificial agents in an interoperable framework.

Research on training cognition with complex tasks, such as video games, may benefit from and contribute to the broad view of cognitive control. The final part of the thesis presents a profile of cognitive control and its generalization based on cognitive training studies, in particular how it may be improved by using action video game training. More specifically, we contrasted the brain connectivity profiles of people that are either habitual action video game players or do not play video games at all. We focused in particular on brain networks that have been associated with cognitive control. Our results show that cognitive control emerges from a distributed set of brain networks rather than individual specialized brain networks, supporting the view that action video gaming may have a broad, general impact of cognitive control. These results also have practical value for cognitive scientists studying cognitive control, as they imply that action video game training may offer new ways to test cognitive control theories in a causal way.

Taken together, the current work explores a variety of approaches from within cognitive science disciplines to contribute in novel ways to the fascinating and long tradition of research on cognitive control. In the age of ubiquitous computing and large datasets, bridging the gap between behavior, brain, and computation has the potential to fundamentally transform our understanding of the human mind and inspire the development of intelligent artificial agents.


\newpage