\pagenumbering{roman}

\section*{Abstract}

Cognitive control is essential to human functioning, as it allows us to adapt and respond to a wide range of situations and environments. The possibility to enhance cognitive control in a way that transfers to real life situations could greatly benefit individuals and society.

However, the lack of a formal, quantitative definition of cognitive control has limited progress in developing effective cognitive control training programs. To address this issue, the first part of the thesis focuses on gaining a clearer understanding of cognitive control and how to measure it. This is accomplished through a large-scale text analysis that integrates cognitive control tasks and related constructs into a cohesive knowledge graph. This knowledge graph provides a more formal definition of cognitive control based on previous research, which can be used to guide the study of cognitive control. The second part of the thesis aims at computational understanding of cognitive control, in particular to study what features of the task (i.e., the environment) and what features of the cognitive system (i.e., the agent) determine cognitive control, its functioning, and generalization. The thesis first presents a reinforcement learning environment where artificial agents can be directly compared to humans in a variety of cognitive tests, and second, a computational framework that functionally decouples general cognitive control in human or artificial agents. The proposed framework is a flexible, differentiable end-to-end deep learning model that separates the act of control from the controlled act, and can be trained to perform the same cognitive tests used in cognitive psychology to test humans. Together, the proposed cognitive environment and agent architecture further enable comparing human data and artificial agents in an interoperable environment.

Research on the effects of complex tasks, such as video games, on cognitive training may benefit from and contribute to the broad view of cognitive control. The final part of the thesis presents an empirical profile of cognitive control and its generalization, in particular how it may be improved by using action video game training. More specifically, we studied neural determinants of playing action video games in brain networks that are often attributed to cognitive control. Results show that cognitive control emerges from a distributed set of brain networks rather than individual brain networks, supporting a broad, general view of cognitive control. The results have also practical value for cognitive scientists studying cognitive control, as they imply that action video game training may be a new tool for causally studying cognitive control.

Taken together, the current work explores approaches from a variety of cognitive science disciplines that aim to better understand the concept of cognitive control. In the age of ubiquitous computing and large datasets, bridging the gap between behavior, brain, and computation has the potential to fundamentally transform our understanding of the human mind and inspire the development of intelligent artificial agents.

\newpage